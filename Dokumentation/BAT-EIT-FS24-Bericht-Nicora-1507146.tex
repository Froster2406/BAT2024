\documentclass[12pt]{article}

\usepackage{amsmath}

\usepackage{graphicx}

\usepackage{hyperref}

\usepackage{xcolor}

\usepackage{amsmath} % for 'pmatrix' environment

\usepackage{listings} %code representation

\usepackage{trfsigns} %transformationszeichen

\usepackage{float} %floating images

\usepackage{subfig} %images next to one another

\usepackage{amssymb} %arrows

\usepackage{pdfpages} %add pdf to latex file

\renewcommand*\contentsname{Inhaltsverzeichnis} % rename table of content

% colored \pm
\usepackage{stackengine,xcolor}
\def\cpm{\mathbin{\ensurestackMath{\abovebaseline[-3.4pt]{%
				\stackunder[-3.5pt]{\color{green!70}+}{\color{red}-}}}}}

\usepackage[utf8]{inputenc}


\begin{document}
	\begin{titlepage}
		\centering
		\vspace{1cm}
		{\scshape\LARGE Hochschule Luzern \\ Technik und Architektur \par}
		\vspace{1cm}
		{\scshape\Large Bachelor Thesis\par}
		\vspace{1.5cm}
		{\huge\bfseries Entwicklung einer PCB zur Analyse von Umgebungslärm\par}
		\vspace{2cm}
		{\Large\itshape Stefano Nicora\par}
		
		\vfill
		
		% Bottom of the page
		{\large 7. Juni 2024\par}
	\end{titlepage}
	\tableofcontents
	
	\newpage
	\section{Einleitung}
	
	
	
\end{document}