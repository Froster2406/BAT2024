\documentclass[12pt]{article}

\usepackage{amsmath}

\usepackage{graphicx}

\usepackage{hyperref} %querreferenzierung

\usepackage{xcolor}

\usepackage{amsmath} % for 'pmatrix' environment

\usepackage{listings} %code representation

\usepackage{trfsigns} %transformationszeichen

\usepackage{float} %floating images

\usepackage{subfig} %images next to one another

\usepackage{amssymb} %arrows

\usepackage{pdfpages} %add pdf to latex file

\usepackage{graphicx} %list of figures

\usepackage{tablefootnote}

\renewcommand*\contentsname{Inhaltsverzeichnis} % rename table of content

\renewcommand{\listfigurename}{Abbildungsverzeichnis}
\renewcommand{\listtablename}{Tabellenverzeichnis}

\usepackage{fancyhdr} %header and footer

% colored \pm
\usepackage{stackengine,xcolor}
\def\cpm{\mathbin{\ensurestackMath{\abovebaseline[-3.4pt]{%
				\stackunder[-3.5pt]{\color{green!70}+}{\color{red}-}}}}}

\usepackage[utf8]{inputenc}

\begin{document}
	\begin{titlepage}
		\centering
		\vspace{1cm}
		{\scshape\LARGE Hochschule Luzern \\ Technik und Architektur \par}
		\vspace{1cm}
		{\scshape\Large Bachelor Thesis\par}
		\vspace{1.5cm}
		{\huge\bfseries Entwicklung einer PCB zur Analyse von Umgebungslärm\par}
		\vspace{2cm}
		{\Large\itshape Stefano Nicora\par}
		
		\vfill
		
		% Bottom of the page
		{\large 7. Juni 2024\par}
	\end{titlepage}
	\tableofcontents
	
	%to do: add footer and header
	
	\newpage
	\section{Einleitung}
	\subsection{Ausgangslage} \label{Ausgangslage}
	Die Firma hEar hat es sich zum Ziel gesetzt, gegen \color{red}TODO\color{black}. Dazu wurde in der Masterarbeit von Sophie Mia Willener eine Marktanalyse durchgeführt, sowie ein erster Prototyp gebaut. Dieser Prototyp ist jedoch noch unhandlich und nicht für den Massenmarkt geeignet. 
	\subsection{Ziele} \label{Ziele}
	Das Ziel dieser Arbeit ist es, in einem ersten Schritt, auf Basis des vorhandenen Prototypen, ein funktionales, kompaktes und portables Schalldruckpegel-Messgerät zu entwickeln. Dabei sollen folgende Rahmenbedingungen zwingend eingehalten werden:
	\begin{itemize}
		\item Die Laufzeit des Gerätes soll mindestens 12 Stunden betragen.
		\item Das Gerät wird mit einem Akku betrieben. Dieser wird via eines USB-C-Anschlusses aufgeladen.
		\item Der Schalldruckpegel wird mit einem MEMS-Mikrofon aufgezeichnet.
		\item Die Messdaten werden in regelmässigen Abständen auf dem Gerät gespeichert.
		\item Das Gerät verfügt über eine BLE-Schnittstelle um die Messdaten drahtlos an ein Zielgerät zu übertragen.
		\item Der aktuelle Schalldruckpegel wird auf der Vorderseite des Gerätes visuell dargestellt.
	\end{itemize}
	In einem zweiten Schritt, wird das Gerät kalibriert und dessen Qualität mit auf dem Markt bereits vorhandenen Geräten verglichen.
	
	\newpage
	\section{Mikrofon}
	\subsection{Grundlagen}
	\subsubsection{MEMS}
	\color{red}TODO\color{black}
	\subsubsection{I2S} \label{I2S}
	\textbf{I}nter-\textbf{I}ntegrated \textbf{S}ound (I²S) bezeichnet eine Bus-Schnittstelle, welche von Philips zur Übertragung von digitalen Audiosignalen entwickelt wurde. Ähnlich wie I²C (Inter-Integrated Circuit) wird die Schnittstelle jedoch nur innerhalb des Gerätes verwendet. Dabei werden drei Pins zwischen Sender (hier das Mikrofon) und dem Empfänger (hier der Mikrocontroller) benötigt:
	\begin{itemize}
		\item \textbf{SCK} \quad (Serial Clock) \\
		Generiert die Taktrate, welche gleichzeitig die Datenrate der Übertragung definiert. Die Taktrate wird vom Master (hier der Mikrocontroller) vorgegeben.
		\item \color{green}\textbf{WS}\color{black} \quad (Word Select) \\
		Gibt vor, welcher Audiokanal (R, L) übertragen werden soll. Dies ermöglicht es, entweder ein Stereo-Signal oder zwei Mono-Signale wie zum Beispiel zwei Mikrofone zu übertragen.
		\item \color{red}\textbf{SD}\color{black} \quad (Serial Data) \\
		Beinhaltet den eigentlichen Datenstream mit der Datenrate definiert durch \textbf{SCK} und der Länge definiert durch \color{green}\textbf{WS}\color{black}.
	\end{itemize}
	\begin{figure}[H]
		\centering
		\includegraphics[width=1\linewidth]{images/BAT_I2S}
		\caption[]{Übersicht I2S-Signale}
		\label{fig:bati2s}
	\end{figure}
	\subsubsection{PDM} \label{PDM}
	\textbf{P}ulse \textbf{D}ensity \textbf{M}odulation oder Pulsdichtemodulation bezeichnet die Darstellung eines analogen Signals in der digitalen Ebene. Dabei entspricht das Verhältnis zwischen der Anzahl der digitalen "1" und digitalen "0" der Amplitude des analogen Signales. Das PDM-Signal weist eine vielfach schnellere Takrate (meist Faktor 32 / 64), als das ursprünglich abgetastete Mikrofonsignal auf. Dadurch wird auftretendes Rauschen in einen höher gelegenen Frequenzbereich verschoben, welcher bei der anschliessenden Signalverarbeitung mit einem Tiefpassfilter abgeschnitten wird. Im Gegensatz zu I²S (\ref{I2S}) werden lediglich zwei Pins zwischen Mikrofon und Empfänger benötigt:
	\begin{itemize}
		\item \color{red}DAT \color{black} \quad (Data) \\
		Beinhaltet den als PDM-Codierten Datenstream.
		\item CLK \quad (Clock) \\
		Generiert die Taktrate, mit welcher die Daten übertragen werden.
	\end{itemize}
	\begin{figure}[H]
		\centering
		\includegraphics[width=1\linewidth]{images/BAT_PDM}
		\caption{Beispiel PDM-Signal}
		\label{fig:batpdm}
	\end{figure}
	\subsubsection{PCM}
	Bei \textbf{P}ulse \textbf{C}ode \textbf{M}odulation wird jeder analoge Messwert in einem digitalen Bereich abgebildet (Abbildung \ref{fig:batpcm}). Je nach benötigter Genauigkeit fällt dieser Bereich mal grösser, mal kleiner aus. Bei der Datenübertragung mittels I²S (\ref{I2S}) geschieht diese Umwandlung direkt im Sender und erzeugt somit das SD-Signal. Bei Sendern mit PDM-Ausgang (\ref{PDM}) ist dies nicht der Fall. Hier ist im Empfänger zwingend eine Peripherie notwendig, welche das PDM-Signal mittels Unterabtastung in ein PCM-Signal umwandelt.
	\begin{figure}[H]
		\centering
		\includegraphics[width=1\linewidth]{images/BAT_PCM}
		\caption{Beispiel PCM-Signal}
		\label{fig:batpcm}
	\end{figure}
	\subsubsection{Schalldruckpegel}
	\color{red}TODO\color{black}
	\subsubsection{Frequenzbewertung} \label{Frequenzbewertung}
	Unser Gehör reagiert nicht nur nicht linear auf Lautstärkeveränderungen, sondern zusätzlich auch unterschiedlich Sensitiv auf unterschiedliche Frequenzen im hörbaren Bereich (20Hz-20kHz). Insbesondere im Bereich zwischen 3kHz bis 4kHz \cite{impairments_basics_2004} ist das Gehör besonders anfällig auf äussere Stimulation. Abbildung \ref{fig:phon-curve}, ursprünglich durch Fletcher \& Munson, mittlerweilen auch durch den ISO226-Standard beschrieben, zeigt diesen Umstand. Um der Problematik aus Abbildung \ref{fig:phon-curve} sorge tragen zu können, wird ein sogenannten Frequenzbewertungsfilter \footnote{ Weighting curves}, Abbildung \ref{fig:frequenzbewertung}, eingesetzt. Dabei werden heutzutage nur noch die dB(A), sowie der dB(C)-Filter eingesetzt. Die beiden Filter unterscheiden sich folgendermassen:
	\begin{itemize}
		\item dB(A) \\
		Reproduziert die 40 Phon Kurve \\
		Widerspiegelt das menschliche Gehör besser, als der dB(C)-Filter
		\item dB(C) \\
		Reproduziert die 100 Phon Kurve.\\
		Filtert weniger tiefe und hohe Frequenzen als dB(A) heraus\\
		Wird verwendet, um besonders laute Umgebungen zu messen
	\end{itemize}
	\begin{figure}[H]
		\centering
		\includegraphics[width=1\linewidth]{images/phon-curve}
		\caption{Phon-Kurven im Vergleich $\vert$ Quelle: \cite{impairments_basics_2004}}
		\label{fig:phon-curve}
	\end{figure}
	\begin{figure}[H]
		\centering
		\includegraphics[width=1\linewidth]{images/Frequenzbewertung}
		\caption{Frequenzbewertungsfilter $\vert$ Quelle: \cite{noauthor_frequenzbewertung_2023}}
		\label{fig:frequenzbewertung}
	\end{figure}
	
	
	\subsection{Komponentenwahl}
	\subsubsection{Kriterien}
	\begin{itemize}
		\item Standort der Mikrofonöffnung
		\item Ausgangssignal
		\item Maximaler Schalldruckpegel
		\item MEMS-Technologie
		\item Frequenzbereich
	\end{itemize}
	\subsubsection{Vergleich}
	\begin{table}[!ht]
		\centering
		\begin{tabular}{|l|l|l|l|l|l|l|}
			\hline
			\textbf{Typ} & \textbf{Protokoll} & \textbf{Öffnung} & \textbf{SNR} & \textbf{AOP[dB]} & \textbf{$f_{min}$ [Hz]} & \textbf{$f_{max}$[Hz]} \\\hline
	ICS-43434 & I2S & Unten & 65 & 120 & 60 & 20000 \\ \hline
	ICS-41351 & PDM & Oben & 65 & 120 & 50 & 20000 \\ \hline
	MMICT5838-00-012 & PDM & Unten & 68 & 133 & 27 & 20000 \\ \hline
	CMM-4737DT-26386-TR & PDM & Oben & 58 & 120 & 100 & 10000 \\ \hline
	CMM-4737DT-26186-TR & PDM & Oben & 58 & 120 & 100 & 10000 \\ \hline
	MP34DT05(-A) & PDM & Oben & 64 & 122.5 & 100 & 10000 \\ \hline
	MP34DT06J & PDM & Oben & 64 & 122.5 & 100 & 10000 \\ \hline
	SPH0645LM4H-B & I2S & Unten & 65 & 120 & 20 & 10000 \\ \hline
	\end{tabular}
	\caption{Vergleich Mikrofone}
	\label{table:vergleich-mikrofone}
\end{table}

	\subsubsection{Fazit}
	$\frac{1}{3}$ der untersuchten Mikrofone weisen die Mikrofon-Öffnung auf der Unterseite auf. Dies ermöglicht es, einen höheren SNR gewährleisten zu können, jedoch ist diese Montageart für das Endprodukt nicht geeignet (siehe Kapitel \ref{Entwicklung}). Bei den restlichen Mikrofonen sticht das ICS-41351 von TDK InvenSense mit seinem hohen SNR, AOP sowie der hohen Frequenzbandbreite heraus. Dadurch hat dieses das Rennen gewonnen.
	\newpage
	\section{Mikrocontroller}
	\subsection{Grundlagen}
	\subsubsection{DMA}
	Bei \textbf{D}irect \textbf{M}emory \textbf{A}ccess handelt es sich um eine Art Steuerbaustein, welcher unabhängig und parallel zum eigentlichen Rechenkern arbeitet. Dieser hat, wie der Rechenkern selbst, Zugriff auf den gesamten Datenbus. Dieser ermöglicht es, effizient Daten von Peripherien wie der UART, I²S, SPI, ect. in den RAM und umgekehrt zu laden. Dadurch wird der Rechenkern entlastet und kann für wichtigere Aufgaben eingesetzt oder gar in einen Energiesparmodus versetzt werden. Dabei gibt es zwei Möglichkeiten mit dem DMA-Baustein zu interagieren:
	\begin{itemize}
		\item Bus Master \\
		Der DMA-Baustein wird einmalig durch den Rechenkern aufgesetzt und läuft danach autonom. So können beispielsweise via I²S einkommende Daten automatisch aus dem Eingangsregister in den Speicher geladen werden und bei genügend grosser Datenmenge (definiert durch den Entwickler), kann Rechenkern wieder aus dem Energiesparmodus aufgeweckt werden.
		\item Programmed I/O \\
		Jeder Arbeitsprozess des DMA-Bausteins wird durch den Rechenkern gestartet. Da sich DMA und Rechenkern den Datenbus teilen, wird so garantiert, dass die Speicheroperation zum gewünschten Zeitpunkt stattfindet.
	\end{itemize}
	\subsubsection{Bootloader}
	Unter einem Bootloader versteht man Code, welcher grundsätzlich nach dem erstmaligen flashen, persistent im System bleibt. Er dient als Ausgangspunkt für die eigentliche Software und ermöglicht es beispielsweise auch, während der Laufzeit des Gerätes, Softwareupdates herunter zu laden und diese anschliessend einzuspielen. Insbesondere wenn Funkprotokolle wie Bluetooth oder WLAN eingesetzt werden, ist ein Bootloader oftmals unerlässlich. Der Nachteil ist jedoch, dass dieser den ohnehin knappen Speicherplatz weiter reduziert.
	\subsubsection{RTC}
	Unter einer \textbf{R}eal \textbf{T}ime \textbf{C}lock versteht man ein Stück Hardware (intern oder extern), welche die Zeit ab Inbetriebnahme misst. Übliche RTCs haben, aufgrund ihrer Bauweise eine Genauigkeit von 100ppm. Da dies einer Zeitverschiebung von 4min und 19s @32kHz entspricht, enthalten die meisten RTCs Kompensationsmechanismen, um diese Zeit auf unter 100ms zu bringen. \cite{dighe_tps65950_2008}
% 	https://www.ti.com/lit/an/swca024/swca024.pdf?ts=1709375495514&ref_url=https%253A%252F%252Fwww.google.com%252F
	\subsubsection{Low Power}
	Mit Low Power wird, insbesondere bei Embedded-Systemen, die reduzierte Leistungsaufnahme beschrieben. Oftmals verfügen die Systeme über mehrere Modi mit jeweils unterschiedlichen Eigenschaften. Das Ziel ist es, mit jedem tieferen Modus etwas mehr Energie zu sparen und somit die Akkulaufzeit zu verlängern. Wie in Abbildung  \ref{fig:batlow-power-modestm} ersichtlich, sinkt zwar die Leistungsaufnahme mit jeder Stufe, jedoch werden dabei gewisse Peripherien abgeschaltet und die allgemeine Startzeit des Systems wird erhöht.
	\begin{figure}[H]
		\centering
		\includegraphics[width=0.7\linewidth]{images/BAT_Low-Power-Mode_STM}
		\caption{Vergleich Low-Power-Modi STM $\vert$ Quelle: \cite{noauthor_stm32l4_2024}}
		\label{fig:batlow-power-modestm}
	\end{figure}
	
	\subsubsection{Peripherie in Hardware oder Software}
	Bei allen gängigen Kommunikationsprotokollen ist die Frage, wie diese implementiert werden sollen. Dabei stehen folgende Möglichkeiten zur Auswahl:
	\begin{itemize}
		\item in \textbf{Hardware} \\
		+ Die benötigte Infrastruktur (Register, CRC, etc.) sind bereits physikalisch implementiert. Die Kommunikation kann dementsprechend vom Rechenkern abgekoppelt durchgeführt werden und ermöglicht eine gewisse Parallelität von mehreren Prozessen mit wenig Rechenaufwand. Insbesondere in Verbindung mit DMA, kann so viel Rechenzeit eingespart werden. \\
		- Bei den meisten Mikrocontrollern ist die spezifische Hardware an einzelne Pins gebunden. Dies wirkt sich auf die Flexibilität des PCB-Layouts oder gar des Gesamtsystems (zu wenig Anschlüsse) aus.
		\item in \textbf{Software} \\
		+ Die meisten Pins können flexibel für jedes Protokoll verwendet werden. Dadurch vereinfacht sich auch das PCB-Layouten.\\
		- Bei Verwendung von Ein-Kern-Systemen, welche im Embedded-Bereich mehrheitlich anzutreffen sind, kann keine echte Parallelität von Prozessen erfolgen. Dies verringert die Ausführungsgeschwindigkeit.
	\end{itemize}
	Grundsätzlich ist, nach Möglichkeit und Verfügbarkeit, eine Hardware-Peripherie zu bevorzugen. Insbesondere bei batteriebetriebenen Systemen mit langer Laufzeit, ist diese ausschlaggebend.
	\subsection{Komponentenwahl}
	Es existieren eine Vielzahl von Mikrocontroller-Herstellern. Viele dieser verfügen über eine breite Palette an BLE-tauglichen Chips. Um den geeignetsten darunter zu finden, werden nachfolgend die benötigten Schnittstellen definiert und mehrere, vorselektionierte Mikrocontroller, miteinander verglichen.
	\subsubsection{Kriterien}
	\begin{itemize}
		\item \textbf{I²S} + \textbf{PDM} zur Ansteuerung des Mikrofons.
		\item \textbf{SPI} zur Ansteuerung von möglichen Peripherien wie externe Speicherbausteine oder Displays.
		\item \textbf{I²C} zur Ansteuerung von Laderegler und LED-Treiber-IC.
		\item \textbf{DMA} zur energieeffizienten Verarbeitung der Daten.
		\item \textbf{RTC} um die gemessenen Werten mit einem Zeitstempel verknüpfen zu können.
		\item \textbf{BLE}-Fähiger Mikrocontroller mit vorgefertigter Antenne um die Integration zu vereinfachen.
	\end{itemize}
	\subsubsection{Vergleich}
	\begin{figure}[H]
		\centering
		\includegraphics[width=0.9\linewidth]{tables/BAT_Vergleich-Mikrocontroller}
		\caption{Vergleich von möglichen Mikrocontrollern}
		\label{table:batvergleich-mikrocontroller}
	\end{figure}
	\subsubsection{Fazit}
	Wie dem Vergleich zu entnehmen ist, unterscheiden sich die Mikrocontroller meist nur in der integrierten Flash-Grösse sowie der Leistungsaufnahme bei einer gewissen Taktrate. Dadurch wird die Wahl nicht eingeschränkt, was eine geeignete Evaluation nicht vereinfacht. Ausschlaggebend ist jedoch der Formfaktor. Bis auf Silicon Labs und STmicro muss für einen Mikrocontroller mit vorgefertigter Antenne auf Drittanbieter ausgewichen werden. In Zeiten von Katastrophen wie COVID oder Kriegen inmitten von Handelsrouten ist die Produktverfügbarkeit einer der zentraler Aspekte. Aus diesem Grund wird nachfolgend der EFR32BG22 von Silicon Labs bzw. dessen Modul-Variante, der \textbf{BGM220P}, eingesetzt.
	
	\newpage
	\section{LED}
	Die visuelle Darstellung des Schalldruckpegels kann auf verschiedene Arten umgesetzt werden. Da hEar explizit eine Darstellung mittels LED wünscht, werden nachfolgend die wichtigsten Eigenschaften eingeführt und anschliessend ein passendes Produkt evaluiert.
	\subsection{Grundlagen}
	\subsubsection{Wellenlänge} \label{Wellenlänge}
	Das für den durchschnittlichen Menschen sichtbare Lichtspektrum beginnt bei 380nm (blau) und endet bei 780nm (rot)\cite{noauthor_v-lambda-kurve_2023}. Wie der Abbildung \ref{fig:batv-lambda-curve} zu entnehmen ist, entspricht die Sensitivität einer Gausskurve. Dabei erreicht diese ihr Maximum bei 555nm (grün) bei Betrachtung der Tageskurve. Dementsprechend kann beim Einsatz einer 555nm-LED im Gegensatz zu LEDs mit einer tieferen oder höheren Wellenlänge die zugeführte Leistung bei gleichwertiger Lichtempfindlichkeit reduziert werden.
	\begin{figure}[H]
		\centering
		\includegraphics[width=0.8\linewidth]{images/BAT_v-lambda-curve}
		\caption{Sensitivität des menschlichen Auges auf das Lichtspektrum bei Tag (\color{red}\textbf{schwarz}\color{black}) und Nacht (\color{blue}\textbf{blau}\color{black})$\vert$ Quelle: \cite{noauthor_v-lambda-kurve_2023}}
		\label{fig:batv-lambda-curve}
	\end{figure}
	
	\subsubsection{Leistungsaufnahme}
	\color{red}TODO\color{black}
	\subsubsection{Lichtleistung}
	Neben der Wellenlänge, ist die Lichtleistung ausschlaggebend für die wahrgenommene Helligkeit einer LED. Dabei wird die Lichtleistung in Candela [cd] angegeben, wobei 1 cd der Helligkeit einer Kerze entspricht. Bei einer typischen LED verhält sich die Lichtleistung linear mit dem zugeführten Strom.
	\subsubsection{Vorwiderstand / Strombegrenzung} \label{Vorwiderstand}
	Nach Anlegen einer Spannung, $>=$ Schwellspannung der LED \footnote{variiert je nach LED Typ und Farbe}, wird diese grundsätzlich zu einem Kurzschluss. Um den Stromfluss durch die LED begrenzen zu können, wird normalerweise ein Vorwiderstand verwendet. Der Nachteil des Vorwiderstandes ist es, dass, gegeben durch das ohmsche Gesetzt, über diesem elektrische Leistung in Wärme umgewandelt wird. Bei einem akkubetriebenen Gerät ist dies unerwünscht. Als Alternative kann ein buck converter, Abbildung \ref{fig:batbuck-converter}, oder eine Konstantstromquelle eingesetzt werden. Dadurch entfällt der Vorwiderstand und die Effizienz des Teilsystems wird erhöht.
	\begin{figure}[H]
		\centering
		\includegraphics[width=1\linewidth]{images/BAT_Buck-Converter}
		\caption{Buck Converter}
		\label{fig:batbuck-converter}
	\end{figure}
	
	\subsection{Komponentenwahl}
	\subsubsection{Kriterien}
	\begin{itemize}
		\item Lichtleistung 
		\item Wellenlänge
	\end{itemize}
	\subsubsection{Vergleich}
	\begin{table}[!ht]
		\centering
		\begin{tabular}{|l|l|l|l|l|}
			\hline
			\textbf{Hersteller} & \textbf{Typ} & \textbf{Farbe} & \textbf{Intensität[mcd]\tablefootnote{bei einem Strom von 2mA}} & \textbf{Spannung [V]} \\ \hline
			SunLED & XZCM2CRK53WA-8VF & Rot & 180 & 2.1 \\ \hline
			SunLED & XZCDGK53W-8VF & Grün & 250 & 3.1 \\ \hline
			SunLED & XZCFBB53W-8VF & Blau & 30 & 3 \\ \hline
			Worldsemi & WS2018 & Rot & 55 & 2.2 \\ \hline
			Worldsemi & WS2018 & Grün & 110 & 3.1 \\ \hline
			Worldsemi & WS2018 & Blau & 20 & 3.4 \\ \hline
			RND & 135-00184 & Rot & 7 & 1.6 \\ \hline
		\end{tabular}
		\caption{Vergleich LED}
		\label{table:vergleich-led}
	\end{table}
	\subsubsection{Fazit}
	Wie im Kapitel \ref{Wellenlänge} ersichtlich, ist die Abwägung zwischen Leistungsaufnahme der LED und der effektiven Photonenabgabe nicht trivial. Aus technischer Sicht macht dementsprechend eine LED mit einer Wellenlänge von 555nm (Grün) am meisten Sinn. Nach Rücksprache mit dem Industrieparter, hEar, dient das Gerät an erster Stelle als Warngerät. Warnungen werden üblich mit einer Wellenlänge von 780nm (Rot) gekennzeichnet. Dadurch wird im Endprodukt die Low-Power-LED XZCM2CRK53WA-8V von SunLED eingesetzt.
	\newpage
	\section{Entwicklung}\label{Entwicklung}
	Die im Unterkapitel \ref{Ziele} definierten Ziele sollen nun umgesetzt werden. Dazu soll die von aussen sichtbare Vorderseite möglichst wenig Bauteile aufweisen, um das visuelle Design nicht zu beeinträchtigen. Mit einer angestrebten Gesamtgrösse von 60mm Durchmesser, können die gewählten Bauteile geradenoch auf dem PCB in einer symmetrischen Art platziert werden. 
	\subsection{Konzept}
	\paragraph{Front}
	Die Vorderseite weist folgende Bauteile auf:
	\begin{itemize}
		\item 8 LEDs
		\item Mikrofon
		\item LED-Treiber-IC
	\end{itemize}
	Zudem sind die LEDs sind ringförmig angeordnet, um der Anforderung eines visuellen Ladebalkens entsprechen zu können.
	\paragraph{Back}
	Die Rückseite hingegen, beinhaltet folgende Bauteile:
	\begin{itemize}
		\item USB-C-Ladeanschluss
		\item Polyfuse (Automatisch rückstellbarer Überstromschutz)
		\item Switch um generell das Gerät ein- und auszuschalten
		\item BMS, um den Akku sicher auf- und entladen zu können
		\item Mikrocontroller, welcher sich aufgrund seiner Antenne am Rand des PCBs befinden muss
		\item Akkuhalter
		\item Optionaler Speicherbaustein
	\end{itemize}
	\begin{figure}[H]
		\centering
		\includegraphics[width=0.9\linewidth]{images/BAT_Konzept_Gerät}
		\caption{Konzept PCB-Layout}
		\label{fig:batkonzeptgerat}
	\end{figure}
	\begin{figure}[H]
		\centering
		\includegraphics[width=0.9\linewidth]{images/BAT_Konzept_Softwareablauf}
		\caption{Konzept Softwareablauf}
		\label{fig:batkonzeptsoftwareablauf}
	\end{figure}
	
	\subsection{Hardware}
	\subsubsection{PCB}
	Das PCB durchlief mehrere Iterationen. Nun folgend die Änderungen mit den benötigten Zusatzinformationen wieso, welche Bauteile eingesetzt werden.
	\paragraph{V1-1}\mbox{}\\
	Lediglich kleinere Verdrahtungs- und Platzierungsoptimierungen.
	\paragraph{V1-2}\mbox{}\\
	Folgende Anpassungen wurden vorgenommen:
	\begin{itemize}
		\item Akku-Verpolungsschutz mittels \nameref{P-Kanal MOSFET}.
		\item Testpads für I²C, SPI und PDM entfernt.
		\item USB-C 16 Pin Buchse durch 6 Pin Buchse (power only) ersetzt.
		\item Rotation der LEDs um 40° im Gegenuhrzeigersinn, um die Reihenfolge der LEDs zu gewährleisten. Zusätzliche Optimierung der Platzierung der LEDs, um die Verdrahtung zu vereinfachen.
		\item TRIG-Pin des LED-Treiber-ICs auf Masse geschaltet.
		\item Spannungsteiler zur Messung der Speisespannung via USB-C.
		\item Optionaler Pullup-Widerstand an DRV\_EN um den Anschluss nicht immer manuell schalten zu müssen.
	\end{itemize}
	\paragraph{Sicherung}\mbox{}\\
	Der Fall eines Kurzschlusses auf Geräteseite wird mittels einer rückstellbaren Sicherung (PPTC\footnote{\textbf{P}olymeric \textbf{P}ositive \textbf{T}emperature \textbf{C}oefficient, automatisch rückstellbare Sicherung von Littlefuse}) der Stromfluss unterbrochen. Die gewählte Sicherung blockiert bei einem Strom von 400mA komplett und setzt sich automatisch wieder zurück, sobald die Speisung abgetrennt wird.
	\paragraph{LED-Treiber} \mbox{}\\
	Die Aufgabe des Treibers ist es, den LED-Strom möglichst Energieeffizient zu regeln (siehe \ref{Vorwiderstand}). Der gewählte Treiber ermöglicht es, alle 8 LEDs individuell in 100uA-Schritten zu regeln. So kann die Leistungsaufnahme des Gesamtsystems kontrolliert werden.
	\paragraph{Akku-Laderegler $\vert$ BMS} \mbox{}\\
	Das \textbf{B}attery \textbf{M}anagement \textbf{S}ystem wurde aufgrund folgender, benötigter Eigenschaften gewählt:
	\begin{itemize}
		\item Möglichst wenige externe Bauteile
		\item Über- und Unterspannungsschutz
		\item Überstromschutz
		\item Keine direkte Ansteuerung mittels Mikrocontroller nötig
		\item Möglichkeit zur visuellen Darstellung des Ladestatus
		\item Unterstützung von Li-Po und Li-Ion Akkus
	\end{itemize}
	\paragraph{Spannungsregler} \mbox{}\\
	Der Akku weist eine maximale Nennspannung von 4.2V (typ. 3.6V) auf. Alle gewählten Halbleiter erlauben jedoch maximal eine Speisespannung von 3.6V. Deswegen wird ein möglichst effizienter und kostengünstiger Spannungsregler verwendet. Wie auch bereits das BMS, weist der Spannungsregler möglichst wenig externe Bauteile auf.
	\paragraph{USB-C-Buchse}\mbox{}\\
	Die in V1-0 verwendete Buchse (16 Pin) wird durch eine 6 Pin (power only) ersetzt. Dies, da die ursprünglich geplante Möglichkeit, den Mikrocontroller via USB zu programmieren, wieder verworfen wurde. Dadurch kann eine günstigere Buchse verwendet werden.
	\paragraph{P-Kanal MOSFET} \label{P-Kanal MOSFET} \mbox{}\\
	Für den Verpolungsschutz kann im Grunde auch eine simple Diode (oder Schottky-Diode\footnote{Die Schwellspannung entspricht nur ca. 0.2V}) verwendet werden. Der daraus folgende Spannungsabfall ist jedoch für ein akkubetriebenes Gerät nicht ideal. Die Lösung bietet ein P-Kanal-MOSFET, wie in Abbildung \ref{fig:batverpolungsschutz}. Die Z-Diode, sowie der Gate-Widerstand, werden jedoch weggelassen, da die Batteriespannung die maximale Source-Gate-Spannung des gewählten FETs nicht überschreitet.
	\begin{figure}[H]
		\centering
		\includegraphics[width=0.7\linewidth]{images/BAT_Verpolungsschutz}
		\caption{Verpolungsschutz mit P-Kanal MOSFET}
		\label{fig:batverpolungsschutz}
	\end{figure}
	
	\subsubsection{Kosten}
	Die Tabelle \ref{table:materialliste} beinhaltet alle anfallenden Materialkosten exkl. Versand. Zudem wurde \href{https://jlcpcb.com/}{JLCPCB} als PCB-Hersteller gewählt, da zum einen die Qualität bei 2-Layer-PCBs ausreicht und zum anderen mit der Anbindung an \href{https://www.lcsc.com/}{LCSC} die Bestückung der PCBs direkt im selben Werk stattfindet. Dadurch verkürzt sich die Fertigung des Endproduktes. \\
	Die vorhandenen Kondensatoren, sowie die benötigten Widerstände werden in der Tabelle \ref{table:materialliste} unter Diverse zusammengenommen. Dies aus dem Grund, dass keines dieser Bauteile einer besonderen Toleranz unterliegt. Dadurch kann bei der PCB-Bestellung eine kosten-optimierte Wahl getroffen werden.
\begin{table}[!ht]
	\centering
	\begin{tabular}{|l|l|l|l|}
		\hline
		\textbf{Hersteller} & \textbf{Typ} & \textbf{Bezeichnung} & \textbf{Preis/100} \\ \hline
		JLCPCB & \href{https://cart.jlcpcb.com/quote?orderType=1\&stencilLayer=2\&stencilWidth=40\&stencilLength=40\&stencilCounts=100}{-} & PCB inkl. Versand & 0.394 \\ \hline
		Texas Instruments  & \href{https://www.mouser.ch/ProductDetail/Texas-Instruments/BQ21040DBVR?qs=cttFivMKqWyovUuY6Xwwcw\%3D\%3D}{BQ21040DBVR} & Batterie-Management & 0.516 \\ \hline
		Microchip Technology & \href{https://www.mouser.ch/ProductDetail/Microchip-Technology/MIC5504-3.3YM5-TR?qs=U6T8BxXiZAUmVQ5Zs217qQ\%3D\%3D}{MIC5504-3.3YM5-TR}  & 3V3 LDO & 0.095 \\ \hline
		RS PRO & \href{https://ch.rs-online.com/web/p/knopfzellen-akkus/1834296}{LIR2477} & Akku & 4.085 \\ \hline
		Littlefuse & \href{https://www.mouser.ch/ProductDetail/Littelfuse/1210L020WR?qs=PWhpLWeW8wcFhN6lPv0ohQ%3D%3D}{1210L020WR} & Rückstellbare Sicherung & 0.205 \\ \hline
		Texas Instruments  & \href{https://www.mouser.ch/ProductDetail/Texas-Instruments/LP55231SQX-NOPB?qs=HF2YfZwisE8IcIRPR19gTw\%3D\%3D}{LP55231SQX/NOPB}  & 9-Kanal LED-Treiber IC & 1.35 \\ \hline
		SunLED & \href{https://www.digikey.ch/de/products/detail/sunled/XZCM2CRK53WA-8VF/10449794}{XZCM2CRK53WA-8VF} & LED RED CLEAR 2SMD & 0.2072 \\ \hline
		C\&K & \href{https://www.mouser.ch/ProductDetail/CK/JS102011SAQN?qs=LgMIjt8LuD\%252B69bNM9a\%2FozQ\%3D\%3D}{JS102011SAQN}  & Schalter 0.3A@6V & 0.453 \\ \hline
		Keystone Electronics & \href{https://www.mouser.ch/ProductDetail/Keystone-Electronics/1025-7?qs=2eeJ4RqLicExTS\%2FpSQucQQ\%3D\%3D}{1025-7}  & Batteriehalter 24x7.7mm & 1.26 \\ \hline
		STMicroelectronics & \href{https://www.mouser.ch/ProductDetail/STMicroelectronics/M95P16-IXMNT-E?qs=rQFj71Wb1eV9LflyHvgrDg\%3D\%3D}{M95P16-IXMNT/E}  & 16MBit SPI EEPROM & 0.843 \\ \hline
		Silicon Labs & \href{https://www.mouser.ch/ProductDetail/Silicon-Labs/BGM220PC22WGA2R?qs=7MVldsJ5UayRQss0gz56jA\%3D\%3D}{BGM220PC22WGA2R}  & Mikrocontroller & 6.33 \\ \hline
		TDK InvenSense & \href{https://www.mouser.ch/ProductDetail/TDK-InvenSense/ICS-41351?qs=\%252B6g0mu59x7LDJ1mBnROZzA\%3D\%3D}{ICS-41351}  & MEMS-Mikrofon & 1.05 \\ \hline
		GCT & \href{https://www.mouser.ch/ProductDetail/GCT/USB4125-GF-A?qs=KUoIvG\%2F9IlaIQ4zBJ6gLeA\%3D\%3D}{640-USB4125-GF-A} & USB-C Buchse 6 Pin TOP & 0.381 \\ \hline
		Toshiba & \href{https://www.mouser.ch/ProductDetail/Toshiba/SSM3J334RLF?qs=PiFplXvYe5VBSYhr6TJz8A\%3D\%3D}{SSM3J334R,LF} & P-Channel MOSFET & 0.109 \\ \hline
		Diverse & Capacitor \& Resistor & Abmessung: 0603 [Inch] & 0.6 \\\hline
		~ & ~ & ~ & \textbf{17.90 CHF} \\ \hline
	\end{tabular}
	\caption{Materialliste $\vert$ Stand: 01.04.2024}
	\label{table:materialliste}
\end{table}
	\subsection{Software}
	Im Kern besteht die Software, wie in Abbildung \ref{fig:batkonzeptsoftwareablauf} gezeigt, aus einem einzelnen, sich wiederholenden Kreis. Durch den einfach gehaltenen Ablauf, wird auf den Einsatz eines RTOS verzichtet.
	\subsubsection{Timer}
	\color{red}TODO\color{black}
	\subsubsection{Filterdesign}
	Wie im Unterkapitel \ref{Frequenzbewertung} erläutert, reagiert unser Gehör nicht auf alle Frequenzen gleich. Um diese Frequenzbewertung in die Schalldruckpegel-Berechnung einfliessen zu lassen, wird ein digitales, diskretes Filter benötigt. Die benötigten Frequenzbewertungsfilter werden jedoch ausschliesslich in der S-Ebene (Laplace-Bereich, zeitkontinuierlich) veröffentlicht. Dadurch müssen diese mit der bilinaren Transformation \cite{oppenheim_alan_v_zeitdiskrete_nodate} vom Frequenz- in den Zeitbereich transformiert werden.
	\begin{equation}\label{G_A_s}
		G_A(s) = \frac{7.39705 \cdot 10^9 \cdot s^4}{(s+129.4)^2\cdot (s+676.7)\cdot (s+4636)\cdot (s+76617)^2}
	\end{equation}
		\begin{equation}\label{G_C_s}
		G_C	(s) = \frac{5,91797 \cdot 10^9 \cdot s^2}{(s+129,4)^2\cdot (s+76617)^2}
	\end{equation}
	
	\subsubsection{PDM}
	\color{red}TODO\color{black}
	\subsubsection{I²C}
	\color{red}TODO\color{black}
	\subsubsection{SPI}
	\color{red}TODO\color{black}
	\subsubsection{RTC}
	\color{red}TODO\color{black}
	
	\newpage
	\section{Messungen}
	\subsection{Leistungsaufnahme}
	\color{red}TODO\color{black}
	\subsection{Mikrofon-Kalibrierung}
	\color{red}TODO\color{black}
	\subsection{Vergleich}
	\color{red}TODO\color{black}
	
	\newpage
	\section{Stolpersteine}
	RTC, BLE, Filterdesign, Debugger
	
	\newpage
	\section{Fazit und Ausblick}
	\color{red}TODO\color{black}
	
	\newpage
	\section{Anhang}
	\bibliography{Sources} 
	\bibliographystyle{ieeetr}
	\listoffigures
	\subsection{Fremdwörter}
	CRC \\
	Register \\
	Pins \\
	Embedded \\
	ppm \\
	flashen \\
	Li-Po \\
	Li-Ion \\
	RTOS
	
	
	
\end{document}